%% LyX 2.1.3 created this file.  For more info, see http://www.lyx.org/.
%% Do not edit unless you really know what you are doing.
\documentclass[english]{article}
\usepackage[T1]{fontenc}
\usepackage[latin9]{inputenc}
\usepackage{babel}
\begin{document}
\begin{quotation}
Requirement Analysis Specification Document

Matteo Maria Fusi

Matteo Locatelli
\end{quotation}
\pagebreak{}\tableofcontents{}

\pagebreak{}


\section{Document Overview}


\subsection{Document Purpose}

This document is the Requirements Analysis Specification Document
(RASD) for mytaxi, the system that must be developed. The purpose
of this document is to describe the requirements of this system (both
functional and non functional), to identify its main goals and users
that will mainly interact with it, especially its stakeholders. This
document will also analyze the properties of the domain in which the
system will be used and the constraints that have to be respected.
The main application scenarios will be analyzed to clearly explain
how the system works. This document is intended for developers and
other people that will work on this system and need to know the fundamental
information about it.


\subsection{Document Structure}

Part 1 talks about what is Requirements Analysis Specification Document
(RASD) and how this paper is structured. Part 2 contains a little
introduction of mytaxy service and general information such as the
scope of the service, identified stakeholders and the terminology
used. The Section 3 is composed by the analysis of the requirements,
from the domain properties to functional and non functional requirements.
Section 4 will describe use cases and will be proposed some common
scenarios to the reader. 


\section{The mytaxi service}


\subsection{Scope }

The purpose of the project is to develop a web-based service that
manages the taxi service. Using their smartphone, or a web-app, users
can request a taxi ride in a simple way. Also taxi drivers will use
their smartphone for accept or reject requests of performance. 


\subsubsection{Goals}

The main goals of the proposed service will be: I. Allow customers
to make a reservation. II. Allow customers to make a request. III.
Allow customers to use the system via the web-app version. IV. Allow
customers to use the system via the smartphone app version. V. Guarantee
a fair management of taxi queues for each taxi zone. VI. Allow taxi
drivers to use a mobile application to access to the system. VII.
Allow taxi drivers to accept a request. VIII. Allow taxi drivers to
refuse a request. IX. The service must be extendible.


\subsection{Stakeholders Identification}

The financial stakeholder of this project is probably the taxi company,
that wants to use a system to improve its service in order to satisfy
its customers and to increase the number of people who use its taxi.
Another possible stakeholder is the government of the city, that wants
to employ mytaxi service for the same reasons explained above. The
only difference is that the taxis are administrated by the government
and not by a private company. Citizens or tourist of a city can also
be stakeholders. 


\subsection{Glossary}
\begin{description}
\item [{Customer/Passenger:}] a citizen who requires a taxi ride using
the service.
\item [{Confirmation:}] after a request is taken in charge by a taxi driver,
the system generates a confirmation that will be sent to the customer.
The confirmation informs the customer about the waiting time and the
code of the taxi. mytaxy: is the service proposed in this document. 
\item [{Performance:}] (Taxi related) in this paper is synonym of the term
Ride. 
\item [{Request:}] when a customer requests for a ride to the system, this
will generate a request that will be forwarded to a taxi driver, containing
information about the customer (such as his/hers name and surname
provided during the registration to the system, etc.) and the origin
of the ride. 
\item [{Reservation:}] when a customer wants to book a ride, the system
will generate a reservation instance. The reservation must be made
at least 2 hours before the ride, but the request is processed by
the system only 10 minutes before the performance takes place. Despite
of the request, the customer must also provide to the system the destination
of the ride 
\item [{Ride:}] performance made by a taxi driver that take a customer
from a place to the desired destination. 
\item [{Driver:}] (taxi related) the one who drives, and also usually owns,
a taxi. He will use the service to find customers. 
\item [{Zone:}] a zone is a section which the city is divided in. Zones
are 2 km2 wide.
\end{description}

\section{Requirement Analysis}


\subsection{Domain Properties}
\begin{enumerate}
\item The City is divided in taxi zones. 
\item Zones are 2 km2 wide. 
\item The position of a taxi is known by receiving GPS coordinates from
the smartphone that the taxi driver owns.
\item A taxi is associated to a zone if its location is inside the zone's
boundaries.
\end{enumerate}

\subsection{Assumptions}

This section will clarify some points that aren't explained in the
specification document. 
\begin{enumerate}
\item Every taxi driver should have installed the mytaxi application on
their smartphone for the sake of establish the position of the vehicle
and, so, assign him to a zone of the city. 
\item We assume that taxi drivers will accept or refuse a request in an
admissible time, for example 1 minute.
\end{enumerate}

\subsection{Functional Requirements}


\subsubsection{Overview}

Passengers can use an app installed on their phone or a web-app to
request or reserve a taxi ride. Taxi drivers will also use the same
app to inform the system about their availability and to confirm requests
made by customers. A customer can send a request for a performance
to the system; the system will forward the request to the first taxi
available. The taxi driver can accept or refuse the request: in the
case of rejection the system will contact the second available taxi
and so on. When the request is accepted by a driver, the customer
is notified with the waiting time and the code of the taxi. If a customer
wants to book a ride, it can also make a reservation. If there no
available taxi in the zone, the system will try to contact the three
nearest taxis in the city outside the inquired zone. 


\subsubsection{Analysis}
\begin{enumerate}
\item Allow customers to make a request.

\begin{enumerate}
\item A customer can make a request specifying his/her name and the origin
of the ride.

\begin{enumerate}
\item If the customer is using the smartphone app, he can also specify the
origin of the ride using the GPS.
\end{enumerate}
\item When the request is accepted by a taxi driver a confirm message will
sent to the customer.

\begin{enumerate}
\item If an available taxi is found, the customer is notified with the code
of the taxi and the estimated waiting time.
\item If the available taxi is in a different zone from the one in which
the customer is, when the passenger receive the confirmation he can
decide to accept or refuse the ride.
\end{enumerate}
\end{enumerate}
\item Allow customers to make a reservation.

\begin{enumerate}
\item A customer can make a reservation specifying the name, the origin
and the destination of the ride and the hour at which he wants to
make it.
\item If the customer is using the smartphone app, he can also specify the
origin of the ride using the GPS.
\item The system will confirm the reservation to the customer.
\item The system will automatically make a request 10 minutes before the
time chosen by the customer.
\end{enumerate}
\item Allow customers to use the system via the web-app version. 

\begin{enumerate}
\item A customer can use the web-app version of the system which will allow
him to access to every functionality of the system. IV. Allow customers
to use the system via the smartphone app version.
\end{enumerate}
\item A customer can use the smartphone app version of the system which
will allow him to access to every functionality of the system.
\item Guarantee a fair management of taxi queues for each taxi zone. 

\begin{enumerate}
\item When a taxi enters a zone, if it's available it's added to the queue
of that zone.
\item When a taxi exits a zone, it is removed from the queue of that zone.
\item If a taxi driver refuses a request from a customer, the taxi is moved
to the last place of the queue.
\item If there isn't any available taxi in the customer's zone or every
taxi in it refuses his request, the request will be forwarded to taxis
in other zones. 
\end{enumerate}
\item Allow taxi drivers to use a mobile application to access to the system.

\begin{enumerate}
\item Taxi drivers are given a code and a password, thanks to them they
can log into the system. 
\item Using the app, taxi drivers can choose to appear available to make
a ride.
\item Taxi drivers can also decide to appear unavailable.
\end{enumerate}
\item Allow taxi drivers to accept a request. 

\begin{enumerate}
\item Taxi drivers can accept a request from a customer.
\end{enumerate}
\item Allow taxi drivers to refuse a request.

\begin{enumerate}
\item Taxi drivers can refuse a request from a customer.
\end{enumerate}
\item The service must be extendible. 

\begin{enumerate}
\item Guarantee programmatic interfaces to enable the development of additional
services.
\end{enumerate}
\end{enumerate}

\subsection{Non Functional Requirements}
\begin{enumerate}
\item The UX should be the same in every system in which the application
is available. For example, the user interface should be the same in
IOS and Android apps. 
\item The system must maintain a reasonable time to manage information.
Which means, the time of processing a request made by a customer (the
time elapsed from the submission to the forwarding of the requesto
to a taxi driver) should resolve in no more than 30 seconds. The same
amount of time must not pass between the acceptance or rejection of
the request and the arrival of the confirmation message to the customer.
\end{enumerate}

\subsection{Constraints}
\begin{enumerate}
\item Internet connection is required for accessing the service.
\item The reservation must be made at least 2 house before the time of the
ride.
\item If the customer is using the system with the web-app version, the
origin of the ride must be inserted manually.
\item The GPS of the smartphone used by taxi drivers must be always turned
on, in this way the system knows always the position of the taxis.
\end{enumerate}

\section{Use Case Analysis}
\end{document}
