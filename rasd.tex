%% LyX 2.1.3 created this file.  For more info, see http://www.lyx.org/.
%% Do not edit unless you really know what you are doing.
\documentclass[english]{article}
\usepackage[T1]{fontenc}
\usepackage[latin9]{inputenc}
\usepackage{array}
\usepackage{graphicx}

\makeatletter

%%%%%%%%%%%%%%%%%%%%%%%%%%%%%% LyX specific LaTeX commands.
%% Because html converters don't know tabularnewline
\providecommand{\tabularnewline}{\\}

\makeatother

\usepackage{babel}
\begin{document}
\begin{quote}
Requirement Analysis Specification Document

Matteo Maria Fusi

Matteo Locatelli
\end{quote}
\pagebreak{}\tableofcontents{}

\pagebreak{}


\section{Document Overview}


\subsection{Document Purpose}

This document is the Requirements Analysis Specification Document
(RASD) for mytaxi, the system that must be developed. The purpose
of this document is to describe the requirements of this system (both
functional and non functional), to identify its main goals and users
that will mainly interact with it, especially its stakeholders. This
document will also analyze the properties of the domain in which the
system will be used and the constraints that have to be respected.
The main application scenarios will be analyzed to clearly explain
how the system works. This document is intended for developers and
other people that will work on this system and need to know the fundamental
information about it.


\subsection{Document Structure}

Part 1 talks about what is Requirements Analysis Specification Document
(RASD) and how this paper is structured. Part 2 contains a little
introduction of mytaxy service and general information such as the
scope of the service, identified stakeholders and the terminology
used. The Section 3 is composed by the analysis of the requirements,
from the domain properties to functional and non functional requirements.
Section 4 will describe use cases and will be proposed some common
scenarios to the reader. 


\section{The mytaxi service}


\subsection{Scope }

The purpose of the project is to develop a web-based service that
manages the taxi service. Using their smartphone, or a web-app, users
can request a taxi ride in a simple way. Also taxi drivers will use
their smartphone for accept or reject requests of performance. 


\subsubsection{Goals}

The main goals of the proposed service will be:
\begin{enumerate}
\item Allow customers to make a reservation. 
\item Allow customers to make a request.
\item Allow customers to use the system via the web-app version.
\item Allow customers to use the system via the smartphone app version.
\item Guarantee a fair management of taxi queues for each taxi zone.
\item Allow taxi drivers to use a mobile application to access to the system.
\item Allow taxi drivers to accept a request.
\item Allow taxi drivers to refuse a request.
\item The service must be extendible.
\end{enumerate}

\subsection{Stakeholders Identification}

The financial stakeholder of this project is probably the taxi company,
that wants to use a system to improve its service in order to satisfy
its customers and to increase the number of people who use its taxi.
Another possible stakeholder is the government of the city, that wants
to employ mytaxi service for the same reasons explained above. The
only difference is that the taxis are administrated by the government
and not by a private company. Citizens or tourist of a city can also
be stakeholders. 


\subsection{Glossary}
\begin{description}
\item [{Customer/Passenger:}] a citizen who requires a taxi ride using
the service.
\item [{Confirmation:}] after a request is taken in charge by a taxi driver,
the system generates a confirmation that will be sent to the customer.
The confirmation informs the customer about the waiting time and the
code of the taxi. mytaxy: is the service proposed in this document. 
\item [{Performance:}] (Taxi related) in this paper is synonym of the term
Ride. 
\item [{Request:}] when a customer requests for a ride to the system, this
will generate a request that will be forwarded to a taxi driver, containing
information about the customer (such as his/hers name and surname
provided during the registration to the system, etc.) and the origin
of the ride. 
\item [{Reservation:}] when a customer wants to book a ride, the system
will generate a reservation instance. The reservation must be made
at least 2 hours before the ride, but the request is processed by
the system only 10 minutes before the performance takes place. Despite
of the request, the customer must also provide to the system the destination
of the ride 
\item [{Ride:}] performance made by a taxi driver that take a customer
from a place to the desired destination. 
\item [{Driver:}] (taxi related) the one who drives, and also usually owns,
a taxi. He will use the service to find customers. 
\item [{Zone:}] a zone is a section which the city is divided in. Zones
are 2 km2 wide.
\end{description}

\section{Requirement Analysis}


\subsection{Domain Properties}
\begin{enumerate}
\item The City is divided in taxi zones. 
\item Zones are 2 km2 wide. 
\item The position of a taxi is known by receiving GPS coordinates from
the smartphone that the taxi driver owns.
\item A taxi is associated to a zone if its location is inside the zone's
boundaries.
\end{enumerate}

\subsection{Assumptions}

This section will clarify some points that aren't explained in the
specification document. 
\begin{enumerate}
\item Taxi drivers are inserted in the system from the beginning. They don't
need to make any kind of registration to the service. Credentials
needed for using the app are provided by sending an e-mail to them.
\item A taxi driver can only drive the taxi associated to him/her.
\item Every taxi driver should have installed the mytaxi application on
their smartphone for the sake of establish the position of the vehicle
and, so, assign him to a zone of the city. 
\item We assume that taxi drivers will accept or refuse a request in an
admissible time, for example 1 minute.
\item We assume that the location where the service will be installed and
set up is near the city in which will be exploited.
\end{enumerate}

\subsection{Functional Requirements}


\subsubsection{Overview}

Passengers can use an app installed on their phone or a web-app to
request or reserve a taxi ride. Taxi drivers will also use the same
app to inform the system about their availability and to confirm requests
made by customers. A customer can send a request for a performance
to the system; the system will forward the request to the first taxi
available. The taxi driver can accept or refuse the request: in the
case of rejection the system will contact the second available taxi
and so on. When the request is accepted by a driver, the customer
is notified with the waiting time and the code of the taxi. If a customer
wants to book a ride, it can also make a reservation. If there no
available taxi in the zone, the system will try to contact the three
nearest taxis in the city outside the inquired zone. 


\subsubsection{Analysis}
\begin{enumerate}
\item Allow customers to make a request.

\begin{enumerate}
\item A customer can make a request specifying his/her name and the origin
of the ride.

\begin{enumerate}
\item If the customer is using the smartphone app, he can also specify the
origin of the ride using the GPS.
\end{enumerate}
\item When the request is accepted by a taxi driver a confirm message will
sent to the customer.

\begin{enumerate}
\item If an available taxi is found, the customer is notified with the code
of the taxi and the estimated waiting time.
\item If the available taxi is in a different zone from the one in which
the customer is, when the passenger receive the confirmation he can
decide to accept or refuse the ride.
\end{enumerate}
\end{enumerate}
\item Allow customers to make a reservation.

\begin{enumerate}
\item A customer can make a reservation specifying the name, the origin
and the destination of the ride and the hour at which he wants to
make it.
\item If the customer is using the smartphone app, he can also specify the
origin of the ride using the GPS.
\item The system will confirm the reservation to the customer.
\item The system will automatically make a request 10 minutes before the
time chosen by the customer.
\end{enumerate}
\item Allow customers to use the system via the web-app version. 

\begin{enumerate}
\item A customer can use the web-app version of the system which will allow
him to access to every functionality of the system. IV. Allow customers
to use the system via the smartphone app version.
\end{enumerate}
\item A customer can use the smartphone app version of the system which
will allow him to access to every functionality of the system.
\item Guarantee a fair management of taxi queues for each taxi zone. 

\begin{enumerate}
\item When a taxi enters a zone, if it's available it's added to the queue
of that zone.
\item When a taxi exits a zone, it is removed from the queue of that zone.
\item If a taxi driver refuses a request from a customer, the taxi is moved
to the last place of the queue.
\item If there isn't any available taxi in the customer's zone or every
taxi in it refuses his request, the request will be forwarded to taxis
in other zones. 
\end{enumerate}
\item Allow taxi drivers to use a mobile application to access to the system.

\begin{enumerate}
\item Taxi drivers are given a code and a password, thanks to them they
can log into the system. 
\item Using the app, taxi drivers can choose to appear available to make
a ride.
\item Taxi drivers can also decide to appear unavailable.
\end{enumerate}
\item Allow taxi drivers to accept a request. 

\begin{enumerate}
\item Taxi drivers can accept a request from a customer.
\end{enumerate}
\item Allow taxi drivers to refuse a request.

\begin{enumerate}
\item Taxi drivers can refuse a request from a customer.
\end{enumerate}
\item The service must be extendible. 

\begin{enumerate}
\item Guarantee programmatic interfaces to enable the development of additional
services.
\end{enumerate}
\end{enumerate}

\subsection{Non Functional Requirements}
\begin{enumerate}
\item \textbf{Same UX}: The UX should be the same in every system in which
the application is available. For example, the user interface should
be the same in IOS and Android apps. A similar UX should be reproduced
on the web-app if used by a desktop computer.
\item \textbf{Acceptable performance}: The system must maintain a reasonable
time to manage information. Which means, the time of processing a
request made by a customer (the time elapsed from the submission to
the forwarding of the requesto to a taxi driver) should resolve in
no more than 30 seconds. The same amount of time must not pass between
the acceptance or rejection of the request and the arrival of the
confirmation message to the customer.
\item \textbf{Freeware}: The application must be free to improve its spread.
\end{enumerate}

\subsection{Constraints}
\begin{enumerate}
\item Internet connection is required for accessing the service.
\item The reservation must be made at least 2 house before the time of the
ride.
\item If the customer is using the system with the web-app version, the
origin of the ride must be inserted manually.
\item The GPS of the smartphone used by taxi drivers must be always turned
on, in this way the system knows always the position of the taxis.
\end{enumerate}

\subsubsection{Legal Constraints}
\begin{itemize}
\item Taxis and taxi drivers are subjected by the actual law in the country
where mytaxi is installed and used.
\end{itemize}

\section{Use Case Analysis}


\subsection{Scenarios}


\subsubsection{Scenario 1}

Marcello is a citizen who wants to request a taxi ride to go to work.
He is at home, so he turns on his computer and open the mytaxi website
to call the ride. In order to complete his request, Marcello fills
the form with his name and his address. The system, once it has received
Marcello's request, start looking for a taxi in the same zone of the
customer's house. An available taxi has been found in that zone, so
the request is sent to the smartphone of the driver who is driving
that taxi. The taxi driver decides to accept Marcello's request, so
a confirmation message in sent to the customer with the code of the
taxi and the estimated waiting time. Marcello receives the confirmation
and goes out oh his house to wait for the taxi. Once the taxi arrives
to Marcello's position, the taxi driver verify that Marcello was the
one who made the request and then use his app to appear unavailable.
Marcello tells the driver his destination and the driver takes him
there. At the end of the ride, the customer pays the taxi driver,
who then use again his mytaxi app to appear available.


\subsubsection{Scenario 2}

Oscar is really tired, he always gets lost in the huge city he lives
in. He decides to request a taxi ride using the mytaxi service, so
he downloads the app on his smartphone and fills the form for the
request: he inserts his name e uses the GPS system to detect his position.
The system finds a taxi in the same zone of Oscar, but the taxi driver
is having lunch break so he decides to refuse the request. The request
is then forwarded to the next taxi in the queue and the driver accepts
the call. The taxi driver makes himself appear as unavailable and
then takes Oscar to his house, who pays for the service and thanks
kindly the driver for saving him.


\subsubsection{Scenario 3 }

Gwyn is a foreign traveler who arrives to the city of Milan to visit
it. He decides to request a ride to go back to the hotel, after spending
the whole day walking around. Gwyn downloads the mytaxi app and fill
the request form submitting is name and his position (he doesn't use
the GPS because his smartphone is low on battery). Unfortunately,
there isn't any free taxi in Gwyn's zone, so the system forwards his
request to the nearest taxi in a different zone. The taxi driver accepts
the request, so a confirmation message is sent to the customer. When
Gwyn sees the waiting time he gets very angry, because he has to wait
for more than an hour. He decides to refuse the ride and visit Milan
a little more, waiting for a closer taxi the be avaialble. After half
an hour, Gwyn tries again to make a request and he's lucky this time
because there's a free taxi in his zone. He submits all the needed
information so he can make the ride and reach his hotel.


\subsubsection{Scenario 4}

Jack wants to request a taxi ride, because he has to go to university
but there is a strike and the public transport isn't an option. He
makes a request using the app installed on his smartphone, but the
system can't find any available taxi inside Jack's zone. The request
is sent to the nearest taxi, but the taxi driver refuses the request
because he's stuck in traffic; so the request is forwarded two more
times but the two drivers refuse Jack's call because of the same problem.
Jack is notified that the service is unavailable for the moment, so
he has to decide either not to go to university or wait a little to
make another request and hope that there's some taxi available.


\subsubsection{Scenario 5}

Bob wants to go to visit a friend of his who lives in Bob's city but
far away from his house. Bob decides to take a taxi, then uses the
mytaxi web-app to request a ride. There are three taxi available in
Bob's zone, but all of the taxi drivers refuse the call of the customer
(such things occures very rarely). The request is forwarded to the
nearest available taxi, the driver accept the request and Bob decides
to take that taxi because he doesn't have to wait much time. The driver
takes Bob to his friend's house and Bob pay the driver for the service.


\subsubsection{Scenario 6}

Lautrec of Carim wants to book a request in the city of Lordran, so
he decides to use the mytaxi app that is already installed on his
smartphone. It's 4 pm and Lautrec wants to make a reservation for
5 pm but, because reservations can only be made at least two hours
before the ride, the system notifies him with an error. Lautrec then
decides to postpone is ride and books a ride for 6.30 pm, filling
the form with the hour, his name, the origin and the destination.
10 minutes before 6.30 pm, the system makes a request for Lautrec's
ride and a taxi is found in the same zone of the starting point. At
6.30 pm Lautrec is at the decided place for the start of the ride
and the taxi driver takes him to his destination.


\subsection{Use Cases}


\subsubsection{A customer makes a request}

\begin{tabular}{|c|>{\raggedright}p{10cm}|}
\hline 
Actor & Customer\tabularnewline
\hline 
Goal & Goal number 1\tabularnewline
\hline 
Input condition & The customer is using the web-app or the smartphone app version of
the system.\tabularnewline
\hline 
Event flow & \begin{enumerate}
\item The customer, once he has opened the application, open the form to
make a request.
\item The customer inserts his name into the form.
\item The customer inserts the origin of the ride into the form, manually
or using the GPS system if he's using the app on his smartphone.
\item The customer presses the confirm button.\end{enumerate}
\tabularnewline
\hline 
Output condition & The request has been done correctly and the system starts looking
for an available taxi.\tabularnewline
\hline 
Exception & None.\tabularnewline
\hline 
\end{tabular}

\includegraphics[width=15cm]{\string"/home/fusiled/Desktop/fusi-locatelli-ingsw2/Use cases diagram/make a request\string".eps}


\subsubsection{A customer makes a reservation}

\begin{tabular}{|c|>{\raggedright}p{10cm}|}
\hline 
Actor & Customer, system\tabularnewline
\hline 
Goal & Goal number 2\tabularnewline
\hline 
Input condition & The customer is using the web-app or the smartphone app version of
the system.\tabularnewline
\hline 
Event flow & \begin{enumerate}
\item The customer, once he has opened the application, open the form to
make a reservation.
\item The customer inserts his name into the form.
\item The customer inserts the origin of the ride into the form, manually
or using the GPS system if he's using the app on his smartphone.
\item The customer inserts the destination of the ride into the form.
\item The customer presses the confirm button.
\item The system will notify the customer with a message if the reservation
has been done correctly.\end{enumerate}
\tabularnewline
\hline 
Output condition & Ten minutes before the time of the reserved ride, the system will
look for an available taxi for the customer.\tabularnewline
\hline 
Exception & The customer has to make the reservation at least two house before
the time of te ride, if this condition isn't respected the customer
will be notified with an error.\tabularnewline
\hline 
\end{tabular}

\includegraphics[width=15cm]{\string"/home/fusiled/Desktop/fusi-locatelli-ingsw2/Use cases diagram/make a reservation\string".eps}


\subsubsection{The system searches for an available taxi}

\begin{tabular}{|c|>{\raggedright}p{10cm}|}
\hline 
Actor & System, Taxi driver\tabularnewline
\hline 
Goal & Goal number 5, 7 and 8\tabularnewline
\hline 
Input condition & The customer has made a request or a reservation (in that case the
system does this procedure autmatically ten minutes before the time
of the ride).\tabularnewline
\hline 
Event flow & \begin{enumerate}
\item The system looks for an available taxi inside the customer's zone.
\item If there's an available taxi, the system sends to the taxi driver
the request for the ride.
\item If the taxi driver refuses the request, his taxi is moved to the bottom
of the queue and the request is forwarded to the next taxi in the
queue (that is now at the first place of the queue).
\item When a taxi driver accepts the request, a confirmation message is
sent to the customer with the code of the taxi and the waiting time.\end{enumerate}
\tabularnewline
\hline 
Output condition & \begin{enumerate}
\item The system finds a taxi in the customer's zone that can take the customer
to his destination.
\item The system doesn't find any available taxi in the customer's zone.
\item Every available taxi driver in the customer's zone refuses the customer's
request.\end{enumerate}
\tabularnewline
\hline 
Exception & The taxi driver doesn't answer to the request in the within 1 minute.
In that case it's like the driver has refused the request.\tabularnewline
\hline 
\end{tabular}

\includegraphics[width=15cm]{\string"/home/fusiled/Desktop/fusi-locatelli-ingsw2/Use cases diagram/system searches in zone\string".eps}


\subsubsection{The system searches for an available taxi outside the customer's
zone}

\begin{tabular}{|c|>{\raggedright}p{10cm}|}
\hline 
Actor & System, Taxi driver\tabularnewline
\hline 
Goal & Goal number 5, 7 and 8\tabularnewline
\hline 
Input condition & The system has not found any available taxi in the customer's zone.\tabularnewline
\hline 
Event flow & \begin{enumerate}
\item The system forwards the customer's request to the nearest taxi located
in a zone different from the one in which the customer is.
\item If the taxi driver refuses the call, it's moved to the last position
of the queue it belongs to and the request is sent to the second nearest
taxi to the customer's position.
\item If the second taxi driver refuses the call too, it's moved to the
last position of the queue it belongs to and the request is sent to
the third nearest taxi to the customer's position.
\item If the third taxi driver refuses the call too, it's moved to the last
position of the queue it belongs to and the customer is notified that
the system is temporarly anavailable.
\item If one of the taxi drivers accepts the request, the system sends to
the customer a refusable confirmation message.\end{enumerate}
\tabularnewline
\hline 
Output condition & \begin{enumerate}
\item The system finds a taxi in the that can take the customer to his destination.
\item The system doesn't find a taxi available to take the customer to his
destination.\end{enumerate}
\tabularnewline
\hline 
Exception & None.\tabularnewline
\hline 
\end{tabular}

\includegraphics[width=15cm]{\string"/home/fusiled/Desktop/fusi-locatelli-ingsw2/Use cases diagram/system searches in other zone\string".eps}

\includegraphics[width=15cm]{\string"/home/fusiled/Desktop/fusi-locatelli-ingsw2/Use cases diagram/search second nearest taxi\string".eps}

\includegraphics[width=15cm]{\string"/home/fusiled/Desktop/fusi-locatelli-ingsw2/Use cases diagram/search third nearest taxi\string".eps}


\subsubsection{The system sends a refusable confirmation message}

\begin{tabular}{|c|>{\raggedright}p{10cm}|}
\hline 
Actor & System, customer\tabularnewline
\hline 
Goal & Goal number 1, 2 and 5\tabularnewline
\hline 
Input condition & The system has found an available taxi outside of the customer's zone
and the driver has accepted the call.\tabularnewline
\hline 
Event flow & \begin{enumerate}
\item The system sends a message with the code of the taxi and the waiting
time to the customer.
\item The customer decides to accept or refuse the ride.
\item The taxi driver is notified with the decision of the customer.\end{enumerate}
\tabularnewline
\hline 
Output condition & \begin{enumerate}
\item If the customer accepts the ride, the taxi driver takes him to the
desired destination.
\item If the customer refuses the ride, the request is cancelled.\end{enumerate}
\tabularnewline
\hline 
Exception & None.\tabularnewline
\hline 
\end{tabular}

\includegraphics[width=15cm]{\string"/home/fusiled/Desktop/fusi-locatelli-ingsw2/Use cases diagram/send refusable message\string".eps}
\end{document}
